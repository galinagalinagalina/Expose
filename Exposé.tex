% !TEX TS-program = pdflatex
% !TEX encoding = UTF-8 Unicode

% This is a simple template for a LaTeX document using the "article" class.
% See "book", "report", "letter" for other types of document.

\documentclass[12pt,titlepage=no,parskip=full]{scrartcl} % use larger type; default would be 10pt

\usepackage[utf8]{inputenc} % set input encoding (not needed with XeLaTeX)
\usepackage[ngerman]{babel}

\usepackage[hidelinks]{hyperref}

%%% Examples of Article customizations
% These packages are optional, depending whether you want the features they provide.
% See the LaTeX Companion or other references for full information.

%%% PAGE DIMENSIONS
\usepackage{geometry} % to change the page dimensions
\geometry{a4paper} % or letterpaper (US) or a5paper or....
% \geometry{margin=2in} % for example, change the margins to 2 inches all round
% \geometry{landscape} % set up the page for landscape
%   read geometry.pdf for detailed page layout information

\usepackage{graphicx} % support the \includegraphics command and options

% \usepackage[parfill]{parskip} % Activate to begin paragraphs with an empty line rather than an indent

%%% PACKAGES
\usepackage{booktabs} % for much better looking tables
\usepackage{array} % for better arrays (eg matrices) in maths
\usepackage{paralist} % very flexible & customisable lists (eg. enumerate/itemize, etc.)
\usepackage{verbatim} % adds environment for commenting out blocks of text & for better verbatim
\usepackage{subfig} % make it possible to include more than one captioned figure/table in a single float
% These packages are all incorporated in the memoir class to one degree or another...

%%% HEADERS & FOOTERS
\usepackage{fancyhdr} % This should be set AFTER setting up the page geometry
\pagestyle{fancy} % options: empty , plain , fancy
\renewcommand{\headrulewidth}{0pt} % customise the layout...
\lhead{}\chead{}\rhead{}
\lfoot{}\cfoot{\thepage}\rfoot{}

%%% SECTION TITLE APPEARANCE
\usepackage{sectsty}
\allsectionsfont{\sffamily\mdseries\upshape} % (See the fntguide.pdf for font help)
% (This matches ConTeXt defaults)

%%% ToC (table of contents) APPEARANCE
\usepackage[nottoc,notlof,notlot]{tocbibind} % Put the bibliography in the ToC
\usepackage[titles,subfigure]{tocloft} % Alter the style of the Table of Contents
\renewcommand{\cftsecfont}{\rmfamily\mdseries\upshape}
\renewcommand{\cftsecpagefont}{\rmfamily\mdseries\upshape} % No bold!

%%% END Article customizations

%%% The "real" document content comes below...

\title{Exposé: Kommentarmanagement auf deutschen Nachrichtenseiten}
\subtitle{Abschlussarbeit im Aufbaustudiengang Journalistik}
\author{Nadine Ambrosch}
%\date{} % Activate to display a given date or no date (if empty),
         % otherwise the current date is printed 

\begin{document}
\maketitle




\section{Relevanz des Themas}

Seit es Medien gibt besteht das Bedürfnis, seine Meinung zu Themen und Texten zu
äußern. Nur war das sehr lange sehr schwierig, wenn nicht sogar unmöglich. Die
Leser waren auf das Ermessen der Redaktionen angewiesen, ob und wie ein
Leserbrief zur Veröffentlichung frei gegeben wurde. 

Mit der Kommunikation über
Computer sind diese Zeiten vorbei. Eine dieser neuen Ausdrucksformen dieser
Kommunikation ist die Kommentarfunktion, mit der jeder Nutzer ohne größeren
Aufwand zu einem bestimmten Thema etwas sagen kann. Es ist also geschafft, was
sich sowohl Leser als auch Medienschaffende gewünscht haben. "Journalists’ desire to 
maintain communication with readers has
been historically, and remains today one of the basic tenets of journalism" (Santana, 2011, S. 67).

Der Leser bereichert und beteiligt sich an der journalistischen Arbeit, er nimmt
aktiv am demokratischen Prozess teil, er übernimmt Aufgaben des gate-keepers.
Dadurch, dass man sich jederzeit \glqq einschalten\grqq\ kann, kommt es zu
einer Flut von Äußerungen, deren Qualität nicht garantiert ist. Die Redakteure
sind hin und her gerissen zwischen Nutzen und Nutzlosigkeit, bis hin zu Schaden,
die die Kommentare auf ihren Plattformen mitunter verursachen. Eine Regulierung
ist unumgänglich. Darüber tauchen viele Fragen und Diskussionen unter den
Journalisten zur Organisation von Kommentaren auf.

Das größte Problem stellt der Missbrauch seitens der Kommentierenden dar, die unter dem Schutz
der Anonymität beleidigen und beschimpfen. Das geht mittlerweile soweit, dass
sogar der deutsche Presserat Handlungsbedarf sieht, da sich Beschwerden
mittlerweile zu 60\% auf online erschienene Texte beziehen:

\begin{quote}
„Aber die Zunahme
von Beschwerden, unter anderem zu Leserkommentaren und Online-Archiven, zeigt,
dass wir die Publizistischen Grundsätze an einigen Stellen ergänzen sollten, um
den digitalen Ver\-öffentlichungs\-for\-men besser gerecht zu werden“
Jahrespressekonferenz, 19.02.2014, Presserat. 
\end{quote}


Süddeutsche.de hat sich Mitte des Jahres 2014 für Einschnitte entschieden („Die Pöbler hart weg moderieren“ 
\footnote{\url{http://derstandard.at/2000001936082} vom 12.06.2014 (abgerufen am 19.1.2015)}), um dann die Kommentare ganz weg zu lassen („Sueddeutsche.de schafft die Kommentarfunktion unter Artikeln ab“ \footnote{\url{http://www.taz.de/!145318/} vom
03.09.2014 (abgerufen am 19.1.2015)}), genauso wie reuters.de \footnote{\url{http://derstandard.at/2000007916112} vom 9.11.2014 (abgerufen am 19.1.2015)}.



Wie sieht es bei den anderen deutschen Online-Zeitungen aus? Wie gehen die
Redaktionen mit den Kommentaren um? Lassen sie den Nutzern komplett freie Hand?
Ist eine Anmeldung erforderlich (mit oder ohne Klarnamenpflicht) oder bleiben
die Kommentare anonym? Werden die Kommentare moderiert (vorher oder danach)?
Gibt es Verhaltensrichtlinien für die Nutzer? Wo ist Kommentieren möglich? Gibt
es Verweise auf social media?






\section{Forschungsziel- und fragen}

Mit dieser Arbeit soll der Frage nachgegangen werden, wie das Kommentarmanagement
der größten Nachrichtenportale jeweils aussieht. Es wird eine Bestandsaufnahme
gemacht, wie die Redaktionen Anfang des Jahres 2015 die Kommentare handhaben und
welche Richtlinien dort vorgegeben werden. In diesem Zuge wird ein Überblick des
aktuellen Kommentarmanagements entstehen.

\begin{itemize} \em
  \item FF1: Wie werden die Nutzerkommentare auf deutschen Nachrichtenportalen gehandhabt?
  \item FF2: Wie sieht die Regulierung dieser Nutzerkommentare aus?
  \item FF3: An welche Vorgaben müssen sich die Nutzer halten?
\end{itemize}


\section{Methodische Umsetzung}

Da es sich beim Analysematerial um schriftliche Dokumente handelt, wird zur
methodischen Umsetzung dieses Überblicks eine Dokumentenanalyse, wie sie Künzler
beschreibt (2009, S.159f), herangezogen. Das heißt auch, dass eine qualitative
Inhaltsanalyse gemacht wird, in Abgrenzung zur quantitativen Inhaltsanalyse
(„die [\ldots] ursprünglich zur Analyse grosser Mengen an Medieninhalten entwickelt
wurde“). Nur von qualitativer Inhaltsanalyse zu sprechen lehnt Künzler ab, da es
zu kurz greift bzw.~Mayring (2008) diesen „Begriff (bereits) für drei von ihm
entwickelte Auswertungsverfahren [Zusammenfassung,
Strukturierung/Kategorisierung, Explikation] geprägt hat“.

Hier liefern die jeweiligen Nachrichtenseiten das Material. Dieses Material wird
bestimmten Kategorien, welche auf Grundlage des theoretischen Vorwissen angelegt
wurden, zugeordnet. Die Kategorien müssen vorher genau definiert werden (siehe
Mayring, 2008, deduktive Kategorienanwendung). 

Das deduktive Verfahren wird mit dem induktiven kombiniert, da die theoretischen
Ansätze nicht ausreichend sind. Man setzt dafür empirisch leere Kategorien ein
(Künzler, 2009, S. 163). Diese dienen als „formales Gerüst, das die Konstruktion
empirisch gehaltvoller Kategorien anhand des Datenmaterials ermöglichen soll“
(Kelle/Kluge, 2010, S. 63). 


Bei solchen heuristischen Rahmenkonzepten gibt es also eine Kombination von
theoretischem „Fachwissen des Forschers und seiner Kenntnis des zu
untersuchenden Forschungsfelds“ (Strauss, 1994, S. 65). Es ist eine
Zusammensetzung von zwei Typen von Kategorien (Strauss, 1994, S. 64f), den
soziologisch konstruierten und den natürlichen Kodes. Die soziologisch
konstruierten Kodes entstehen erst im Laufe der Beschäftigung mit dem Material,
die anderen werden vorausgesetzt.



\section{Auswahl der Nachrichtenportale}

Im Sinne der Forschungsfrage und der Dokumentenanalyse wird nicht zufällig auf
irgendwelche Portale zugegriffen. Nicht die statistische Repräsentativität ist
das Ziel der Analyse, sondern das kriteriengesteuerte Sammeln von Dokumenten
(Künzler, 2009).  

Es werden diejenigen Nachrichtenportale ausgewählt, welche die meisten
Besucher verzeichnen können. Diese Quellenbeschreibung ist Teil der
Quellenkritik, die wiederum ein Gütekriterium der Dokumentanalyse ist. Weitere
Bestandteile der Quellenkritik sind: Die Nachrichtenportale sind im Internet
frei verfügbar und für jeden zugänglich (Textsicherung). Die Nachrichtenportale
werden in einem bestimmten Zeitraum betrachtet (äußere Kritik). 

Im August 2014 listet statista.com die die „Top 15 Nachrichtenseiten“ auf
\footnote{\url{http://de.statista.com/statistik/daten/studie/273789/umfrage/reichweite-der-meistbesuchten-nachrichtenwebsites-zielgruppe-ab-10-jahre/)}}.
Diese werden für den vergleichenden Überblick herangezogen und untersucht. Es
handelt sich umfolgende Zeitungen:
 \href{http://www.Bild.de}{Bild.de}, 
 \href{http://www.focus.de}{FOCUS Online}, 
 \href{http://www.spiegel.de}{SPIEGEL ONLINE}, 
 \href{http://www.welt.de}{Die Welt}, 
 \href{http://www.focus.de}{FOCUS Online}, 
 \href{http://www.süddeutsche.de}{süddeutsche.de}, 
 \href{http://www.stern.de}{Stern},   
 \href{http://www.zeit.de}{ZEIT ONLINE}, 
 \href{http://www.FAZ.NET}{FAZ.NET}, 
 \href{http://www.n-tv.de}{n-tv.de},  
 \href{http://www.rp-online.de}{RP ONLINE}, 
 \href{http://www.N24.de}{N24.de},  
 \href{http://www.tagesspiegel.de}{Tagesspiegel}, 
 \href{http://www.Handelsblatt.com}{Handelsblatt.com}, 
\href{http://www.Huffingtonpost.de}{Huffington Post}





\section{Gliederung}

\renewcommand{\labelenumii}{\arabic{enumi}.\arabic{enumii}}
\renewcommand{\labelenumiii}{\arabic{enumi}.\arabic{enumii}.\arabic{enumiii}}
\renewcommand{\labelenumiv}{\arabic{enumi}.\arabic{enumii}.\arabic{enumiii}.\arabic{enumiv}}
\begin{enumerate}
  \item Nutzerbeteiligung im Journalismus
  \item Kommentare als Nutzerbeteiligung
  \begin{enumerate}
    \item Nutzerbeteiligung im Journalismus
    \item Positive Aspekte der Kommentarfunktion
    \begin{enumerate}
      \item Popularität der Interaktion
      \item Öffentlicher Diskurs
      \item Neue Inhalte
      \item Gate-Keeper Funktion
      \item Leserbindung
    \end{enumerate}
    \item Probleme mit den Kommentaren
  \end{enumerate}
  \item Regulierung von Kommentaren (Moderation, Prämoderation, Postmoderation, Anzeigen von Missbrauch, Vorgaben ...)
  \item Dokumentenanalyse (Kategorien, Analyse, ...)
  \item Ergebnis
  \item Fazit
\end{enumerate}





\section{Literatur}

Santana, A. D. (2011). Online Readers' Comments Represent New Opinion Pipeline. Newspaper Research Journal, 32 (3), 66–81

Künzler, M. (2009). Die Liberalisierung von Radio und Fernsehen. Leitbilder der Rundfunkregulierung im Ländervergleich. Konstanz: UVK-Verl.-Ges

Mayring, P. (2008). Qualitative Inhaltsanalyse. Grundlagen und Techniken. 10. Auflage. Weinheim, Basel

Kelle, U./Kluge, S. (1999). Vom Einzelfall zum Typus. Fallvergleich und Fallkontrastierung in der qualitativen Sozialforschung. 
(Qualitative Sozialforschung, Bd. 15) 2. überarbeitete Auflage. Wiesbaden

Strauss, A. (1994). Grundlagen qualitativer Sozialforschung. Datenanalyse und Theoriebildung in der empirischen
soziologischen Forschung. München

Meyer, H. K. \&  Carey, M. C. (2014). In Moderation. Journalism Practice, 8 (2), 213–228. doi: 10.1080/17512786.2013.859838

Santana, A. D. (2014). Controlling the Conversation. Journalism Studies, online first, 1–18. doi: 10.1080/1461670X.2014.972076

Singer, J. B., Hermida, A., Domingo, D., Heinonen, A., Paulussen, S., Quandt, T. et al. (Hrsg.). (2011). Participatory journalism. Guarding open gates at online newspapers. Chichester: Wiley-Blackwell

Singer, J. B. (2014). User-generated visibility: Secondary gatekeeping in a shared media space. New Media \& Society, 16 (1), 55–73. 
doi: 10.1177/1461444813477833

Reich, Z. (2011). „User Comments: The Transformation of Participatory Space.“ In: Participatory
Journalism: Guarding Open Gates at Online Newspapers, Singer (2011), S. 57-118

Hermida, A. (2011). „Mechanisms of Participation. How audience options shape the conversation“ In: Participatory
Journalism: Guarding Open Gates at Online Newspapers, Singer (2011), S. 156-191

Quandt, T. (2011).  „Understanding a New Phenomenon. The significance of participatory journalism“ In: Participatory
Journalism: Guarding Open Gates at Online Newspapers, Singer (2011), S. 140-176

Vujnovic, M. (2011).  „Participatory Journalism in the Marketplace Economic motivations behind the practices“ In: Participatory
Journalism: Guarding Open Gates at Online Newspapers, Singer (2011), S. 122-154

Singer, J. B. (2011).   „Taking Responsibility: Legal and ethical issues in participatory journalism“ In: Participatory
Journalism: Guarding Open Gates at Online Newspapers, Singer (2011), S. 119-138

\end{document}
